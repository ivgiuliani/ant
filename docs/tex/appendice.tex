\begin{chapter}{Appendice}

\begin{section}{Il gioco dell'8}
\label{sec:gioco-filetto}

Il gioco dell'8 consiste nell'ordinare dei tasselli, numerati da 1 ad 8 pi\`u una
casella vuota, in senso orario, in un quadrato di dimensioni 3x3 lasciando la casella
vuota al centro, ovvero portando i tasselli in questa posizione:

\begin{center}
\begin{tabular}{| c | c | c |}
\hline
1 & 2 & 3\\\hline
8 &   & 4\\\hline
7 & 6 & 5\\\hline
\end{tabular}
\end{center}

L'ordinamento pu\`o avvenire spostando i tasselli verso la casella vuota, quindi
i movimenti permessi su un tassello sono solo quelli di uno spostamento a destra,
sinistra, sopra ed in sotto a seconda di dove si trovi la casella vuota.

La configurazione iniziale consiste in un quadrato i cui tasselli sono stati
mischiati, tuttavia non tutte le configurazioni iniziali sono valide, ovvero
non \`e detto che una configurazione generata casualmente sia risolvibile. Un
possibile (e sicuro) metodo di generazione di una configurazione casuale consiste
nel partire dalla configurazione obiettivo e spostare casualmente\footnote{l'idea
di generare configurazioni \textit{casuali} tramite un algoritmo che \`e predicibile
per definizione pu\`o apparire in qualche modo paradossale e difatti lo \`e. Si
rimanda tuttavia l'approfondimento di questo problema a testi specializzati dato
che l'argomento esula dall'oggetto di questa discussione.} i tasselli.
In questo modo la configurazione generata sar\`a sempre risolvibile, infatti
ci sar\`a almeno un cammino che porta alla risoluzione del problema (basta
seguire a ritroso il procedimento che ha portato alla generazione della
configurazione in esame).
\end{section}

\begin{section}{Problem Solving}
\label{sec:problem-solving}
Come gi\`a accennato nel paragrafo \ref{sec:heuristicdef}, il problem
solving comprende una serie di metodi atti a costruire programmi che mirano a raggiungere
obiettivi, spesso indeterminati e incerti. Per raggiungere tale obiettivo \`e necessaria
una descrizione, anche se parziale o incompleta, di una situazione presente e di una
situazione desiderata. Queste situazioni possono essere rappresentate con schemi (o
linguaggi) che siano sufficientemente ricchi da permetterci di descrivere entit\`a,
eventi, oggetti o casi (situazioni) ed eventualmente differenze tra coppie di situazioni.
Inoltre \`e necesaria una lista di operatori, definiti in un linguaggio che fa riferimento
al processo solutivo, che applicati a tali situazioni permettono di ottenerne delle nuove.
In aggiunta, ogni sequenza di operatori pu\`o essere considerata un operatore. In questo
modo, si pu\`o pensare alla soluzione di un problema come ad un operatore (composto) nel
linguaggio di processo che trasforma l'oggetto che descrive la situazione iniziale
nell'oggetto che descrive la situazione desiderata.
\end{section}

\begin{section}{Confronto con CLIPS}
\label{sec:clips-comparison}
Abbiamo voluto mettere a confronto i due sistemi per osservarne le differenze di
comportamento, sia in termini prestazionali che in termini di paradigmi adottati.
Innanzitutto, CLIPS\footnote{http://clipsrules.sf.net} utilizza l'algoritmo RETE
(vedi \cite{rete}) per il pattern matching che \`e estremamente pi\`u efficiente
di quello utilizzato da ANT, oltre ad essere asintoticamente indipendente sia dal
numero di regole che dal numero di fatti presenti (cosa, purtroppo, non vera in ANT).

Il termine di paragone principale \`e stato, come gi\`a pi\`u volte osservato nel
corso di questo scritto, quello del gioco dell'8. CLIPS mette a disposizione pi\`u
strategie di risoluzione di conflitti ma, purtroppo, nessuna delle quali fa parte
della categoria degli algoritmi informati come A* o hill climbing. Una conseguenza
di ci\`o \`e stata che l'unico algoritmo di risoluzione dei conflitti che pu\`o
risolvere il gioco dell'otto in CLIPS \`e stato quello random, ovvero la scelta
di una regola, tra quelle applicabili, a caso. Il problema principale \`e che non
viene messa a disposizione una \textit{memoria} degli stati precedenti (nonostante
sia in realt\`a implementabile esplicitamente), pertanto tutti gli altri algoritmi
testati portano in una sequenza ciclica infinita il programma, come ad esempio lo
spostamento in circolo del tassello vuoto (es: sx, poi su, poi dx, poi gi\`u).

Inoltre, non \`e definito esplicitamente uno stato finale (ma ne \`e definito
uno iniziale). Il controllo di raggiungimento della combinazione vincente deve,
quindi, avvenire sotto forma di regola che si attiver\`a solo ed esclusivamente
quando i tasselli saranno nella posizione richiesta.

Questo ha comportato una strategia di risoluzione leggermente diversa rispetto a
quella vista in ANT per quanto riguarda il gioco dell'otto con prestazioni
da parte dei due sistemi abbastanza diverse.
Prendendo in esempio due stati iniziali, il primo che richiede un minimo di due
mosse per essere risolto, ed il secondo che ne richiede sette, abbiamo fatto un
confronto sull'esecuzione. Con CLIPS infatti, nonostante la risoluzione sia
palesemente pi\`u veloce, l'algoritmo fornisce una sequenza risolutiva che contiene
al suo interno molte pi\`u mosse del necessario per quanto riguarda il caso della
configurazione di difficolt\`a ``media'' con un minimo di sette mosse (ne fornisce
infatti un numero variabile tra le 17 e 30), riuscendo comunque a risolvere il caso
semplice con esattamente due mosse.
\end{section}

\begin{section}{Grammatica del linguaggio}
\label{sec:grammar}
Segue la notazione in sintassi BNF estesa della grammatica del
linguaggio utilizzato:

\begin{verbatim}
<cifra> := 1|2|3|4|5|6|7|8|9|0

<numero> := <cifra> | <cifra> <numero>

<lettera> := a|b|c|d|e|f|g|h|i|j|k|l|m|n|o|p|q|r|s|t|u|w|x|y|z

<stringa> := '"' <lettera>* '"'

<identificatore> := <lettera> |
                    <numero> |
                    <lettera><identificatore> |
                    <numero><identificatore>

<espressione> := <predicato> |
                 <predicato> 'and' <espressione> |
                 <predicato> 'or' <espressione> |
                 '(' <espressione> ')'

<argomento> := <numero> | <stringa>

<argomenti> := <argomento> | <argomento> ','

<azione> := <nomepredicato>'(' <argomenti> ')' ';' |
            <nomepredicato>'(' <argomenti> ')' ';' <azione>

<regola> := 'beginRule' <identificatore>
               <espressione>
               '>'
               <azione>
            'endRule'

<heuristica> := 'beginHeuristic' <identificatore>
                   <azione>
                'endHeuristic'

<attributi> := (<identificatore> '=' (<stringa>|<numero>) ';')+

<fatto> := 'beginFact' <identificatore>
             <attributi>
           'endFact'

<opzioni> := 'set'
               <attributi>
             'endSet'

<MAIN> := <opzioni> (<fatto>)+ (<regola>)+ (<euristica>)+
\end{verbatim}

\end{section}

\begin{section}{Predicati disponibili}
\label{sec:predicates}

	\begin{subsection}{Predicati per la LHS}
	\label{sec:predicates-lhs}

		\begin{subsubsection}{equals}
		\label{sec:predicates-lhs-equals}
		Sintassi: \verb!equals([attributo], valore1, valore2, ..., valoreN)!\\
		Se non \`e specificato il nome dell'attributo, allora richiede solo due parametri e
		ritorna vero se sono uguali, falso altrimenti. Nel caso in cui sia specificato
		il nome dell'attributo come \verb,attributo,, il quale ha come valore
		\verb![ n1, n2, ..., nM ]!, ritorna vero se e solo se $ N = M $ e
		$ valore_{i} = n_{i} \forall i $.
		\end{subsubsection}

		\begin{subsubsection}{different}
		\label{sec:predicates-lhs-different}
		Sintassi: \verb!different([attributo], valore1, valore2, ..., valoreN)!\\
		Se non \`e specificato il nome dell'attributo, allora richiede solo due parametri e
		ritorna vero se sono diversi, falso altrimenti. Nel caso in cui sia specificato
		il nome dell'attributo come \verb,attributo,, il quale ha come valore
		\verb![ n1, n2, ..., nM ]!, ritorna vero se e solo se $ N \neq M $ oppure
		$ \exists i \ni' valore_{i} \neq n_{i} $.
		\end{subsubsection}

		\begin{subsubsection}{gt}
		\label{sec:predicates-lhs-gt}
		Sintassi: \verb!gt(valore1, valore2)!\\
		Ritorna vero se \verb,valore1, \`e maggiore di \verb,valore2,. Questo
		predicato si applica solo a valori numerici, nel caso di stringhe ritorna
		sempre falso.
		\end{subsubsection}

		\begin{subsubsection}{gte}
		\label{sec:predicates-lhs-gte}
		Sintassi: \verb!gte(valore1, valore2)!\\
		Ritorna vero se \verb,valore1, \`e maggiore o uguale di \verb,valore2,.
		Questo predicato si applica solo a valori numerici, nel caso di stringhe ritorna
		sempre falso.
		\end{subsubsection}

		\begin{subsubsection}{lt}
		\label{sec:predicates-lhs-lt}
		Sintassi: \verb!lt(valore1, valore2)!\\
		Ritorna vero se \verb,valore1, \`e minore di \verb,valore2,. Questo
		predicato si applica solo a valori numerici, nel caso di stringhe ritorna
		sempre falso.
		\end{subsubsection}

		\begin{subsubsection}{lte}
		\label{sec:predicates-lhs-lte}
		Sintassi: \verb!lte(valore1, valore2)!\\
		Ritorna vero se \verb,valore1, \`e minore o uguale di \verb,valore2,.
		Questo predicato si applica solo a valori numerici, nel caso di stringhe ritorna
		sempre falso.
		\end{subsubsection}

		\begin{subsubsection}{within}
		\label{sec:predicates-lhs-within}
		Sintassi: \verb!within(valore, min, max)!\\
		Ritorna vero se \verb,valore, \`e compreso tra \verb,min, e \verb,max, (inclusi),
		ovvero se $ valore \in [ min, max ] $.
		Questo predicato si applica solo a valori numerici, nel caso di stringhe ritorna
		sempre falso.
		\end{subsubsection}

		\begin{subsubsection}{true}
		\label{sec:predicates-lhs-true}
		Sintassi: \verb!true(...)!\\
		Torna sempre vero, indipendentemente da quanti e quali parametri gli vengono 
		passati in input.
		\end{subsubsection}

		\begin{subsubsection}{false}
		\label{sec:predicates-lhs-false}
		Sintassi: \verb!false(...)!\\
		Torna sempre falso, indipendentemente da quanti e quali parametri gli vengono 
		passati in input.
		\end{subsubsection}

	\end{subsection}

	\begin{subsection}{Predicati per la RHS}
	\label{sec:predicates-rhs}

		\begin{subsubsection}{add}
		\label{sec:predicates-rhs-add}
		Sintassi: \verb!add(valore1, valore2)!\\
		Aggiunge a \verb,valore2, a \verb,valore1,. \`E necessario che \verb,valore1, sia una
		variabile per far si che il risultato dell'operazione venga salvato. Inoltre, sia
		\verb,valore1, che \verb,valore2, devono essere valori numerici (o in caso di variabili,
		devono essere gi\`a definite).
		\end{subsubsection}

		\begin{subsubsection}{sub}
		\label{sec:predicates-rhs-sub}
		Sintassi: \verb!sub(valore1, valore2)!\\
		Sottrae a \verb,valore2, a \verb,valore1,. \`E necessario che \verb,valore1, sia una
		variabile per far si che il risultato dell'operazione venga salvato. Inoltre, sia
		\verb,valore1, che \verb,valore2, devono essere valori numerici (o in caso di variabili,
		devono essere gi\`a definite).
		\end{subsubsection}

		\begin{subsubsection}{mul}
		\label{sec:predicates-rhs-mul}
		Sintassi: \verb!mul(valore1, valore2)!\\
		Moltiplica \verb,valore2, per \verb,valore1,. \`E necessario che \verb,valore1, sia una
		variabile per far si che il risultato dell'operazione venga salvato. Inoltre, sia
		\verb,valore1, che \verb,valore2, devono essere valori numerici (o in caso di variabili,
		devono essere gi\`a definite).
		\end{subsubsection}

		\begin{subsubsection}{define}
		\label{sec:predicates-rhs-define}
		Sintassi: \verb!define(nomevariabile, valore)!\\
		Dichiara una nuova variabile chiamata \verb,nomevariabile,, inizializzata col valore e
		tipo di \verb,valore,.
		\end{subsubsection}

		\begin{subsubsection}{undefine}
		\label{sec:predicates-rhs-undefine}
		Sintassi: \verb!undefine(variabile1, variable2, ..., variabileN)!\\
		Cancella le variabili specificate. Se le variabili non esistono le ignora.
		\end{subsubsection}

		\begin{subsubsection}{find}
		\label{sec:predicates-rhs-find}
		Sintassi: \verb!find(attributo, valore1, valore2, ..., valoreN)!\\
		Effettua una \textit{approximate search} sul fatto in analisi. Se \verb,attributo, \`e
		una variabile non precedentemente definita, allora \verb,find, iterer\`a su tutti
		gli attributi all'interno del fatto cercando un match con i valori di tipo non
		variabile passati come parametro. Appena trova un attributo che soddisfa i requisiti,
		valorizza tutte le variabili non risolte con i valori dell'attributo. Si ferma al
		primo attributo che soddisfa la richiesta.
		\noindent Se \verb,attributo, \`e una costante allora la sostituzione verr\`a fatta
		sui valori dell'attributo del fatto di nome \verb,attributo, come spiegato in
		precedenza.
		\end{subsubsection}

		\begin{subsubsection}{insert}
		\label{sec:predicates-rhs-insert}
		Sintassi: \verb!insert(attributo, valore1, valore2, ..., valoreN)!\\
		Inserisce un nuovo attributo nel fatto in esame di nome \verb,attributo, e con i
		valori da \verb,valore1, a \verb,valoreN,.
		\end{subsubsection}

		\begin{subsubsection}{remove}
		\label{sec:predicates-rhs-remove}
		Sintassi: \verb!remove(attributo)!\\
		Rimuove l'attributo \verb,attributo, dal fatto in analisi.
		\end{subsubsection}

		\begin{subsubsection}{edit}
		\label{sec:predicates-rhs-edit}
		Sintassi: \verb!edit(attributo, valore1, valore2, ..., valoreN)!\\
		Sostiuisce i valori di \verb,attributo, con quelli da \verb,valore1, a \verb,valoreN,.
		\end{subsubsection}

		\begin{subsubsection}{equals}
		\label{sec:predicates-rhs-equals}
		Sintassi: \verb!equals([attributo], valore1, valore2, ..., valoreN, risultato)!\\
		Verifica l'uguaglianza di attributi come descritto in \ref{sec:predicates-lhs-equals}
		(\verb,equals,) e salva il risultato dell'operazione nella variabile specificata per
		\verb,risultato, come 1 se l'uguaglianza era valida o come 0 altrimenti.
		\end{subsubsection}

		\begin{subsubsection}{different}
		\label{sec:predicates-rhs-different}
		Sintassi: \verb!different([attributo], valore1, valore2, ..., valoreN, risultato)!\\
		Verifica la non uguaglianza di attributi come descritto in \ref{sec:predicates-lhs-different}
		(\verb,different,) e salva il risultato dell'operazione nella variabile specificata per
		\verb,risultato, come 1 se l'uguaglianza era non valida o come 0 altrimenti.
		\end{subsubsection}

        \begin{subsubsection}{return}
        \label{sec:predicates-rhs-return}
        Sintassi: \verb!return(valore)!\\
        Specifica il valore di ritorno della funzione euristica. Utilizzabile esclusivamente durante
        la definizione di un'euristica (ovvero entro il blocco \verb,beginHeuristic,-\verb,endHeuristic,).
        \end{subsubsection}

	\end{subsection}
\end{section}

\end{chapter}

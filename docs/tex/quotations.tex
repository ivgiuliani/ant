\begin{titlepage}
    \fancyhf{}
    \newpage
        \vspace*{\fill}

        \begin{center}
        \begin{minipage}{0.9\textwidth}
        Se ci capita per le mani qualche volume, per esempio, di teologia  o
        metafisica scolastica, domandiamoci: contiene qualche ragionamento sperimentale
        su questioni di fatto e di esperienza? No. E allora gettiamolo nel fuoco,
        perch\`e non contiene che sofisticherie ed inganni.
        \end{minipage}
        \end{center}

        \begin{flushright}
        \textit{David Hume}
        \end{flushright}

        \vspace*{\fill}

        \begin{center}
        \begin{minipage}{0.9\textwidth}
        Our ultimate objective is to make programs that learn from their experience as
        effectively as humans do. We shall say that a program has common sense if it
        automatically deduces for itself a sufficient wide class of immediate consequences
        of anything it is told and what it already knows.
        \end{minipage}
        \end{center}

        \begin{flushright}
        \textit{John McCarthy, ``Programs with Common Sense''}
        \end{flushright}

        \vspace*{\fill}

        \begin{center}
        \begin{minipage}{0.9\textwidth}
        Overloading + coercizione + inclusione = Mal di testa da C++
        \end{minipage}
        \end{center}

        \begin{flushright}
        \textit{Donato Malerba, ``Metodi Avanzati di Programmazione''}
        \end{flushright}

        \vspace*{\fill}	

    \newpage
\end{titlepage}

